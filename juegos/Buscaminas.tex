\section{Bucaminas}

\begin{figure}[htbp]
\begin{center}
\includegraphics[width=.60\textwidth]{./imagenes/minesweeper.png}
\caption{Buscaminas}
\label{Buscaminas}
\end{center}
\end{figure}
Buscaminas\footnote{\url{http://minesweeperonline.com/}} es uno de los juegos más jugados debido a lo ubicuo de su distribución. Fue incluido en 1992 en la versión de Windows 3.1 y desde entonces lo hemos encontrado presente en todas las versiones de dicho sistema operativo.
En la figura \ref{Buscaminas} puede ver una implementación web del juego.
La premisa del juego es simple: Limpiar el campo de juego sin hacer explotar ninguna de las minas que se encuentran en la cuadrícula.
El tablero puede crecer hasta una dimension de 16x30
\subsubsection{¿Por qué es uno de mis juegos favoritos?}
\begin{itemize}
\item[Javier Tibau] Las reglas del juego son sencillas y fáciles de entender. A pesar de esto, el juego no es atractivo para todo el mundo, creo que es un gusto adquirido. Las reglas me fueron presentadas por mi papá, quien en su máquina de trabajo con Windows 3.11 era uno de los pocos juegos ``divertidos'' que tenía. Para mi, el gran interés del juego es que destaca (o esconde) la resolución de problemas con fondo algebraico. En cierto momento del juego, y para el jugador que ha estudiado álgebra lineal, el reventar una casilla se torna similar a descifrar un sistema de ecuaciones con varias incógnitas. Los sistemas sencillos son bien definidos y tienen 2, 3 o hasta 4 incógnitas, mientras los más complejos pueden inclusive tener múltiples soluciones.
\end{itemize}
